\chapter{Information Retrieval}\label{Information_Retrieval}
Nel contesto dell'era moderna caratterizzata da un'enorme quantità di dati e informazioni disponibili, il campo dell'Information Retrieval (IR) emerge come un pilastro fondamentale per affrontare la sfida di estrarre conoscenza significativa da questa vastità di risorse. L'Information Retrieval si configura come un insieme di metodologie e processi volti a consentire agli individui di individuare, recuperare e accedere alle informazioni rilevanti all'interno di un mare di dati non strutturati.

Attraverso l'Information Retrieval, l'obiettivo primario è consentire agli utenti di navigare efficacemente attraverso l'abbondanza di informazioni, affiancando la loro esigenza di accesso immediato a contenuti rilevanti. L'Information Retrieval si pone come tramite tra gli individui e la vastità di dati disponibili, facilitando la traduzione delle esigenze degli utenti in query comprensibili e restituendo risultati pertinenti e significativi. In questo contesto, l'efficienza e l'accuratezza del processo di recupero dell'informazione diventano essenziali, influenzando direttamente l'esperienza dell'utente e la fruibilità delle risorse informative.

La continua crescita esponenziale delle risorse digitali, dai documenti di testo alle immagini, ai video e oltre, ha ampliato il campo dell'Information Retrieval, richiedendo soluzioni sempre più sofisticate ed efficienti per la ricerca e il recupero dei dati. Questa evoluzione ha portato alla coniugazione dell'Information Retrieval con discipline quali l'Intelligenza Artificiale, l'Apprendimento Automatico e l'Elaborazione del Linguaggio Naturale, che hanno apportato nuove prospettive e approcci innovativi al processo di recupero dell'informazione.

Nel seguito di questo capitolo, esploreremo le radici e l'evoluzione storica dell'Information Retrieval, analizzando come le sfide emergenti stiano plasmando il campo e delineando il panorama attuale in termini di approcci, tecnologie e prospettive future. Attraverso questa analisi, si prefigge l'obiettivo di evidenziare l'importanza cruciale dell'Information Retrieval nell'era moderna e la sua continua trasformazione nel contesto delle mutevoli esigenze informative degli individui.\section{Cos'è l'Information Retrieval} WIP
L'Information Retrieval rappresenta un campo interdisciplinare che si concentra sulla progettazione e lo sviluppo di metodologie, algoritmi e sistemi finalizzati a recuperare informazioni rilevanti da collezioni di dati eterogenei e spesso imponenti. A differenza dell'indicizzazione tradizionale dei dati, che organizza le risorse in base a criteri predefiniti, l'Information Retrieval mira a fornire agli utenti la capacità di formulare query personalizzate e ottenere risultati coerenti con le loro necessità informative.

Questo processo di recupero dell'informazione si basa su una serie di principi fondamentali. In primo luogo, gli utenti esprimono le loro richieste attraverso query, che possono essere composte da parole chiave, frasi o domande complete. L'Information Retrieval si impegna quindi a interpretare e comprendere le intenzioni sottostanti di queste query, cercando di identificare non solo le corrispondenze letterali, ma anche il contesto e il significato implicito. Successivamente, il sistema di Information Retrieval cerca all'interno della collezione di dati e documenti indicizzati per individuare quelli che meglio soddisfano le richieste dell'utente.

L'elemento chiave nell'Information Retrieval è la valutazione della rilevanza. Ogni documento recuperato viene valutato in base alla sua pertinenza rispetto alla query dell'utente. Tuttavia, la rilevanza è spesso un concetto sfumato e soggettivo, poiché può variare in base al contesto, all'utente e alle circostanze. Di conseguenza, molti sistemi di Information Retrieval utilizzano tecniche di ranking per ordinare i risultati in modo da presentare quelli più rilevanti o probabilmente interessanti all'utente in cima alla lista.

Un aspetto cruciale dell'Information Retrieval è la necessità di bilanciare l'efficienza e l'accuratezza. I sistemi di ricerca devono essere in grado di gestire grandi volumi di dati e rispondere rapidamente alle query degli utenti, garantendo al contempo che i risultati siano altamente pertinenti. Questa sfida ha portato allo sviluppo di algoritmi di ricerca e tecniche di indicizzazione sempre più sofisticati, che sfruttano modelli matematici, apprendimento automatico e processamento del linguaggio naturale per migliorare la qualità del recupero delle informazioni.

In sintesi, l'Information Retrieval svolge un ruolo cruciale nella navigazione dell'oceano di dati digitali, fornendo strumenti e metodologie per individuare e accedere alle informazioni desiderate. Questo campo continua a evolversi in risposta alla crescente complessità delle risorse informative e alle mutevoli aspettative degli utenti, spingendo verso l'adozione di approcci sempre più innovativi e tecnologicamente avanzati.

\begin{figure}[H]
    \includegraphics{images/comefunzionair.jpg}
    \caption{Come funziona generalmente un sistema di Information Retrieval}
\end{figure}
\section{La storia dell'Information Retrieval} WIP
La disciplina è nata dall'esigenza di gestire l'enorme quantità di dati disponibili, in particolare nel contesto delle biblioteche, delle raccolte di documenti e,
più recentemente, del mondo digitale e dell'Internet.

Negli Anni '60 e '70 si è vista la crescita delle prime basi di dati elettroniche e dei primi sistemi di IR automatizzati. Il progetto MEDLARS presso la National Library of Medicine negli Stati Uniti è stato uno dei primi esempi di successo nell'applicazione dell'IR alle pubblicazioni mediche. Nel 1965, Gerard Salton ha sviluppato il modello matematico di "Term Frequency-Inverse Document Frequency" (TF-IDF), che è ancora oggi una tecnica chiave nell'IR.

Negli Anni '80 e '90 l'avvento dei computer personali e il boom delle reti informatiche hanno reso possibile l'accesso alle informazioni da remoto. Motori di ricerca come Archie, Gopher e, più tardi, il famoso WebCrawler, hanno iniziato a indicizzare e cercare pagine Web.

Gli Anni 2000 hanno visto la nascita dei motori di ricerca moderni, tra cui Google che ha introdotto algoritmi di ranking più avanzati basati su link e testo. Il concetto di "PageRank" è stato cruciale per migliorare la qualità dei risultati di ricerca.

Negli Anni 2010 l'IR è diventato sempre più legato all'apprendimento automatico e all'intelligenza artificiale. I motori di ricerca hanno iniziato a utilizzare algoritmi di apprendimento per personalizzare i risultati in base al comportamento dell'utente.

Oggi l'IR è diventato una parte fondamentale della nostra vita quotidiana. Motori di ricerca, sistemi di raccomandazione e algoritmi di classificazione guidano il nostro accesso all'informazione. Con l'avvento del Web semantico e dell'elaborazione del linguaggio naturale, l'IR sta evolvendo per comprendere il significato contestuale delle query e dei documenti, portando a risultati di ricerca ancora più raffinati.


In sintesi, la storia dell'Information Retrieval è una progressione che va dall'organizzazione manuale delle informazioni alla creazione di potenti motori di ricerca guidati dall'intelligenza artificiale. Questo campo in continua evoluzione gioca un ruolo cruciale nell'aiutare le persone a trovare e comprendere le informazioni nel mare sempre crescente di dati digitali. Nella figura~\ref{fig:timelineir} si può osservare una schemetizzazione delle ``ere'' dell'information retrieval in cui appunto negli anni '60 si iniziava ad avere le basi teoriche, applicate poi nei due decenni successivi e poi il grande balzo in avanti che c'è stato negli ultimi decenni del secolo scorso per arrivare allo stato attuale in cui si riesce a chiedere tramite query non troppo precise e ottenere comunque buoni risultati.
\begin{figure}
    \includegraphics[width=0.75\pdfpagewidth]{images/timelineir.png}
    \caption{Timeline dell'Information Retrieval} \label{fig:timelineir}
\end{figure} 
\section{I task dell'Information Retrieval} WIP
L'Information Retrieval comprende una serie di compiti interconnessi, ognuno dei quali mira a migliorare il processo di recupero delle informazioni. Tra questi task vi sono la classificazione dei documenti, l'indicizzazione (ovvero l'associazione di termini chiave ai documenti per facilitarne la ricerca), la rappresentazione dei documenti attraverso modelli matematici, il calcolo della similarità tra query e documenti, e l'ordinamento dei risultati in base alla loro rilevanza. Inoltre, l'Information Retrieval si è esteso alla ricerca multimediale, che coinvolge il recupero di dati come immagini e video.
\section{Lo stato dell'arte} WIP
Attualmente, l'Information Retrieval si trova in una fase di grande innovazione grazie all'uso di algoritmi di apprendimento automatico e di tecniche di intelligenza artificiale. I motori di ricerca moderni utilizzano algoritmi di ranking sofisticati, come PageRank e algoritmi di apprendimento profondo, per fornire risultati di ricerca altamente personalizzati e pertinenti. Inoltre, l'analisi semantica e contestuale gioca un ruolo sempre più importante nell'interpretazione delle query degli utenti e nella comprensione dei documenti. Tuttavia, sfide come la privacy, l'etica e l'interpretazione accurata delle intenzioni degli utenti pongono ancora sfide significative per il campo.

In conclusione, l'Information Retrieval continua a essere un campo dinamico in costante evoluzione, in cui la combinazione di conoscenze informatiche, teorie dell'informazione e metodologie di intelligenza artificiale sta plasmando il modo in cui le persone accedono e interagiscono con le informazioni nell'era digitale.