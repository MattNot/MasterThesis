L'evoluzione dell'Information Retrieval è intrecciata con lo sviluppo della tecnologia dell'informazione e delle comunicazioni. Dai primi sistemi basati su schede perforate e indici cartacei, si è passati gradualmente a sistemi digitali sempre più sofisticati. Uno dei punti di svolta storici è stato l'avvento dell'informatica, che ha reso possibile l'automatizzazione dei processi di recupero attraverso algoritmi e modelli matematici. Dagli anni '60 in poi, la crescita di Internet e il World Wide Web hanno ulteriormente trasformato il panorama dell'Information Retrieval, portando a sfide come l'eccesso di informazioni e la necessità di algoritmi di ranking sempre più avanzati.