Il processo di fine tuning, noto anche come adattamento o addestramento aggiuntivo, è una fase cruciale nell'addestramento dei modelli di linguaggio come i Large Language Model (LLM). Consiste nell'adattare un modello preaddestrato su un compito specifico o su un insieme di dati particolare. Questo processo consente al modello di acquisire conoscenze specifiche e dettagliate su un determinato dominio o compito senza dover addestrare il modello da zero.

Ecco come avviene il processo di fine tuning in dettaglio:

Preparazione dei dati: Prima di iniziare il processo di fine tuning, è necessario raccogliere e preparare un insieme di dati specifico per il compito di interesse. Questi dati dovrebbero essere etichettati correttamente per consentire al modello di apprendere la relazione tra gli input e le uscite desiderate.

Caricamento del modello preaddestrato: Si parte da un modello preaddestrato, che è stato addestrato su un ampio corpus di testi generici. Questo modello ha già acquisito una conoscenza generale del linguaggio e delle strutture grammaticali.

Definizione del compito: Si specifica il compito che si desidera affrontare durante il fine tuning. Ad esempio, il compito potrebbe essere la classificazione di testi, la generazione di testo in uno specifico stile o la traduzione automatica.

Aggiunta degli strati finali: Si aggiungono uno o più strati di rete neurale al modello preaddestrato. Questi strati aggiuntivi sono specifici per il compito in questione. Ad esempio, nel caso della classificazione di testi, gli strati finali potrebbero essere costituiti da un livello completamente connesso che trasforma l'output del modello in una previsione di classe.

Inizializzazione dei pesi: Gli strati aggiuntivi vengono inizializzati casualmente o utilizzando pesi provenienti dal modello preaddestrato. Questo passaggio è importante perché consente di iniziare l'addestramento con una base già buona acquisita durante il preaddestramento.

Addestramento: Si addestra il modello sui dati del compito specifico utilizzando l'ottimizzazione del gradiente stocastico (SGD) o altri algoritmi di ottimizzazione. Durante l'addestramento, il modello aggiusta i pesi degli strati aggiuntivi per adattarsi ai dati del compito, cercando di minimizzare la perdita tra le previsioni del modello e le etichette di training.

Validazione e tuning degli iperparametri: Si monitora il modello durante l'addestramento utilizzando un insieme di dati di validazione separato. Questo permette di valutare le prestazioni del modello su dati non visti e di regolare gli iperparametri, come il tasso di apprendimento o le dimensioni del batch, per migliorare le prestazioni.

Controllo dell'overfitting: Come nell'addestramento di modelli di deep learning, è importante prevenire l'overfitting durante il fine tuning. Questo può essere fatto utilizzando tecniche di regolarizzazione come l'eliminazione casuale, la riduzione del tasso di apprendimento e l'uso di dati di validazione.

Test finale: Dopo che il modello ha raggiunto una buona performance sui dati di validazione, viene testato su un insieme di dati di test separato, che non è stato utilizzato né per il preaddestramento né per il fine tuning. Questo test fornisce una stima delle prestazioni del modello su dati completamente nuovi.

Deployment: Una volta che il modello ha superato con successo il test finale, può essere deployato per l'uso nel mondo reale. Può essere integrato in applicazioni, sistemi automatizzati o piattaforme che richiedono il compito specifico che il modello è stato addestrato a svolgere.

In sintesi, il fine tuning è un passaggio essenziale per adattare un modello preaddestrato a compiti specifici. Consente di sfruttare la conoscenza generale del linguaggio acquisita durante il preaddestramento e di personalizzare il modello per affrontare compiti più specializzati o dominii specifici.