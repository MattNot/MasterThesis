GPT-3 è la terza iterazione della serie di modelli di linguaggio GPT sviluppata da OpenAI. È stato rilasciato nell'estate del 2020 ed è noto per la sua straordinaria capacità di generazione di testi altamente coerenti e coerenti su una vasta gamma di argomenti.

\subsubsection{Architettura e Dimensione}
GPT-3 è basato sull'architettura del transformer, che si è dimostrata altamente efficace nell'elaborazione del linguaggio naturale. Tuttavia, ciò che distingue GPT-3 è la sua incredibile dimensione: il modello è stato allenato su 175 miliardi di parametri, rendendolo uno dei modelli di linguaggio più grandi e complessi mai creati.

\subsubsection{Zero-Shot, Few-Shot, and One-Shot Learning}
Una delle caratteristiche sorprendenti di GPT-3 è la sua capacità di apprendimento con pochissimi esempi di input. GPT-3 può eseguire il cosiddetto "zero-shot learning", dove viene fornita un'istruzione generica (ad esempio, "Traduci questa frase in francese") e il modello genera l'output corrispondente. Inoltre, GPT-3 può eseguire il "few-shot learning", dove vengono forniti solo pochi esempi di input e il modello continua la generazione coerente. Infine, c'è anche il "one-shot learning", dove GPT-3 può apprendere da un singolo esempio e generare output coerenti.

\subsubsection{Applicazioni e Creatività}
GPT-3 ha una vasta gamma di applicazioni. Può essere utilizzato per scrivere articoli, generare codice di programmazione, rispondere a domande, creare conversazioni fluide, tradurre testi e persino per creare arte e musica. L'ampia versatilità di GPT-3 è ciò che lo rende così affascinante e potenzialmente utile in molti campi.

\subsubsection{Limitazioni e Preoccupazioni}
Anche se GPT-3 è impressionante nella sua capacità di generazione di testi, presenta alcune limitazioni. Può produrre contenuti che sembrano coerenti ma possono essere inaccurati o fuorvianti. Inoltre, il modello non ha una vera comprensione del mondo come gli esseri umani e può occasionalmente generare affermazioni assurde o non corrette. Ciò solleva preoccupazioni riguardo alla diffusione di informazioni errate o alla manipolazione dei contenuti.

\subsubsection{Accesso tramite API}
OpenAI ha reso GPT-3 accessibile tramite un'API (interfaccia di programmazione delle applicazioni) a sviluppatori e aziende. Questo ha reso possibile l'integrazione delle capacità di GPT-3 in una varietà di applicazioni e servizi.

\subsubsection{Discussione etica e regolamentazione}
L'enorme potenziale di GPT-3 ha sollevato discussioni su questioni etiche e regolamentari. C'è il timore che il modello possa essere utilizzato per la diffusione di disinformazione, per la creazione di contenuti dannosi o per altre attività maliziose. Ciò ha portato OpenAI e la comunità a riflettere sul modo migliore per garantire un utilizzo responsabile e sicuro delle capacità di GPT-3.

\subsection[ChatGPT e GPT4]{ChatGPT e GPT-4}

ChatGPT è un chatbot presentato nel novembre del 2022 da OpenAI, è stato addestrato a partire dai modelli Instruct GPT o GPT-3.5 di OpenAI, che sono l'evoluzione dei modelli di GPT-3. Gli Instruct GPT (come code-davinci-002, text-davinci-002, text-davinci-003) sono modelli in cui il pre-addestramento è stato ottimizzato manualmente da addestratori umani. Nello specifico ChatGPT è stato sviluppato da un GPT-3.5 utilizzando l'apprendimento supervisionato e l'apprendimento per rinforzo. Il 14 marzo 2023 è stata annunciata l'introduzione di GPT-4, un modello multimodale su larga scala che può accettare input di immagini, video, audio e testo e produrre output di testo
Al 2023, ChatGPT è stato il servizio tecnologico che ha raggiunto più velocemente i 100 milioni di utenti, in soli due mesi \cite{chat100}.
Oltre al clamore nell'opinione pubblica ChatGPT ha avuto degli effetti secondari per niente banali sopratutto nel mercato finanziario, è bastato infatti che alcune aziende dicessero di voler integrare ChatGPT nei loro servizi per far schizzare il valore delle azioni alle stelle, le azioni di Buzzfeed, un'azienda di media digitali non legata all'IA, son salite del 120\% \cite{buzz120} dopo aver affermato di voler adottare ChatGPT nei loro post/articoli.
Uno dei limiti maggiori di chatgpt è che la sua base di conoscenza è ferma al 2021 e solo recentemente OpenAI ha rilasciato modalità con cui poter connettere ad internet il suo modello tramite plugin di terze parti.

\subsubsection{GPT-4}
Si è parlato di GPT-4 appena è stato pubblicato il 3.5, infatti ci si chiedeva cosa sarebbe stato in grado di fare un modello ancora più evoluto.

La distinzione tra GPT-3.5 e GPT-4 può essere difficilmente notabile nelle conversazioni quotidiane. La differenza emerge quando la complessità del compito raggiunge una soglia sufficiente: GPT-4 è più affidabile, creativo e in grado di gestire istruzioni molto più sfumate rispetto a GPT-3.5. 
La differenza più evidente, però, è che con GPT-4 si può parlare anche tramite immagini, testo o immagini e testo insieme.

GPT-4 è il primo modello di intelligenza artificiale ad aver superato entrambe le parti, a scelta multipla e scritta, dell'UBE, l'Uniform Bar Exam, con un punteggio superiore alla media degli esaminati reali \cite{gpt4bar}

Sam Altman, CEO di OpenAI, ha affermato che il training di GPT-4 è costato circa 100 milioni di dollari.