BERT, acronimo di "Bidirectional Encoder Representations from Transformers", è un modello di linguaggio sviluppato da Google AI nel 2018. Si tratta di un tipo di rete neurale basata su trasformer che ha dimostrato un'enorme capacità di comprensione del linguaggio naturale e ha raggiunto risultati di stato dell'arte in varie attività di elaborazione del linguaggio naturale (NLP).

La caratteristica chiave di BERT è la sua capacità di comprensione bidirezionale del contesto. Nei modelli precedenti, come le reti neurali ricorrenti (RNN) o le reti neurali convoluzionali (CNN), l'elaborazione del testo avveniva in una direzione specifica, limitando la capacità del modello di catturare il contesto da entrambi i lati di una parola. BERT, d'altra parte, può considerare il contesto sia a sinistra che a destra di una parola in una frase, permettendo una comprensione più approfondita delle relazioni e del significato.

La storia di BERT è stata segnata da importanti sviluppi:

Annuncio e Rilascio: BERT è stato presentato da Google AI nel 2018 attraverso un articolo di ricerca e il modello preaddestrato è stato rilasciato insieme al codice sorgente. Questo ha permesso a ricercatori e sviluppatori di utilizzare il modello preaddestrato per affrontare varie attività di NLP.

Preallenamento Supervisionato: Prima di essere utilizzato per task specifici, BERT viene preaddestrato su grandi quantità di testo in modo supervisionato. Durante questa fase, il modello impara a prevedere parole mancanti in frasi parziali, consentendogli di acquisire una conoscenza generale del linguaggio.

Fine-Tuning: Dopo il preallenamento, il modello viene sottoposto a un processo di fine-tuning su dati di addestramento specifici per una determinata attività. Questo può essere il riconoscimento dell'entità, la classificazione del testo, la traduzione automatica e così via.

Impatto Significativo: BERT ha ottenuto risultati straordinari su una serie di benchmark di NLP, superando molti dei modelli esistenti in molte attività. Il suo successo ha ispirato ulteriori sviluppi in questo campo.

Derivazioni e Modelli Successivi: Dopo BERT, sono state sviluppate varie varianti e modelli successivi che hanno cercato di affrontare alcune limitazioni e migliorare le prestazioni. Alcuni esempi includono GPT-2, RoBERTa, T5 e altri.

Applicazioni Pratiche: BERT e i suoi discendenti sono stati utilizzati in una vasta gamma di applicazioni, come i motori di ricerca, l'elaborazione automatica del linguaggio naturale, l'analisi dei sentimenti, la risposta alle domande e molto altro.

In sintesi, BERT ha rappresentato un passo significativo nell'ambito della comprensione del linguaggio naturale, dimostrando come l'architettura dei transformer e il preallenamento bidirezionale possano portare a risultati notevoli in una varietà di attività di NLP.