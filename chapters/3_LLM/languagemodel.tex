Un modello di linguaggio è un tipo di modello statistico o neurale progettato per comprendere e generare il linguaggio naturale umano. In altre parole, è un sistema che mira a catturare le regolarità e le strutture del linguaggio in modo da poter produrre testo coerente e comprensibile o effettuare operazioni di analisi e comprensione del testo.

\subsubsection{Comprendere il Linguaggio Naturale}
Comprendere il linguaggio naturale è una delle sfide più complesse nell'ambito dell'elaborazione del linguaggio naturale (NLP). Il linguaggio umano è intrinsecamente ambiguo, dipendente dal contesto e ricco di sottigliezze. I modelli di linguaggio cercano di catturare queste sfumature e strutturare le informazioni in modo che possano essere manipolate e utilizzate per compiere compiti specifici.

\subsubsection{Task di un Modello di Linguaggio}
I task per cui risulta utile un modello di linguaggio sono molteplici, questa n'è una lista non esaustiva:

\begin{itemize}
    \item Incorporamento delle Parole: Un modello di linguaggio parte spesso dall'incorporamento delle parole (word embeddings), che assegna a ogni parola un vettore numerico. Questo permette al modello di rappresentare parole simili come vettori vicini nello spazio, catturando relazioni semantiche.
    \item Struttura delle Frasi: Un modello di linguaggio deve catturare le strutture sintattiche e semantiche delle frasi. Questo coinvolge la comprensione delle dipendenze tra le parole e la capacità di identificare parti del discorso, come nomi, verbi, aggettivi, ecc.
    \item Contesto: La comprensione del contesto è essenziale. Una parola può avere significati diversi in contesti diversi. I modelli di linguaggio cercano di considerare le parole circostanti per determinare il significato corretto.
    \item Generazione del Testo: I modelli di linguaggio possono generare testo coerente partendo da una sequenza di input. Questo può essere utilizzato per generare articoli, rispondere a domande o creare testo creativo.
    \item Classificazione e Analisi: I modelli di linguaggio possono anche essere utilizzati per compiti di classificazione, come l'analisi del sentimento o la categorizzazione dei testi in categorie specifiche.
\end{itemize}

\subsubsection{Approcci per creare un modello di linguaggio}
Ci sono principalmente due approcci nell'implementazione di modelli di linguaggio:

\begin{itemize}
    \item \textbf{Modelli Statistici}: Questi modelli utilizzano metodi statistici tradizionali per modellare la probabilità di una sequenza di parole. I modelli n-gram e gli Hidden Markov Models (HMM) sono esempi di approcci statistici.
    \item \textbf{Reti Neurali}: Gli approcci neurali sono diventati predominanti nell'NLP moderno. Le reti neurali, in particolare le reti ricorrenti (RNN) e i Transformer, sono in grado di catturare relazioni complesse e apprendere rappresentazioni significative dai dati.

\end{itemize}


\subsubsection{Ruolo attuale dei modelli di linguaggio}
Nell'era moderna, i modelli di linguaggio pre-addestrati, come BERT, GPT e altri basati su Transformer, hanno dominato l'NLP. Questi modelli vengono addestrati su enormi quantità di testo e sono in grado di catturare relazioni semantiche e sintattiche complesse. Sono la base di molti successi recenti nell'analisi del linguaggio naturale, dall'analisi del sentimento alla traduzione automatica e molto altro ancora.

In sintesi, un modello di linguaggio è un sistema che mira a comprendere e generare linguaggio naturale umano attraverso metodi statistici o neurali. Questi modelli giocano un ruolo cruciale nell'elaborazione del linguaggio naturale e sono al centro di numerosi avanzamenti nell'NLP moderno.