Addestrare un Large Language Model (LLM) come GPT-3 comporta diversi passaggi complessi. Ecco una spiegazione dettagliata del processo:

Raccolta dei dati: La prima fase coinvolge la raccolta di un vasto insieme di testi da utilizzare come dati di addestramento. Questi testi possono provenire da una varietà di fonti, come libri, articoli, siti web, conversazioni, ecc. È importante avere una vasta gamma di testi per garantire che il modello sia esposto a diversi stili di scrittura, argomenti e voci.

Tokenization: I testi raccolti vengono suddivisi in "token". Un token è l'unità minima di testo, che può essere una singola lettera, una parola o addirittura un pezzo di parola (ad esempio, "chat" potrebbe essere suddiviso in "ch" e "at"). Questo processo è essenziale per gestire in modo efficiente i dati testuali e renderli pronti per l'elaborazione da parte del modello.

Creazione del vocabolario: Dai token ottenuti attraverso la tokenizzazione, viene creato un vocabolario. Questo vocabolario rappresenta l'insieme di tutti i token unici presenti nei dati di addestramento. Ogni token è associato a un ID numerico univoco all'interno del vocabolario.

Embedding dei token: Ogni token nel vocabolario viene associato a un vettore numerico chiamato "embedding". Gli embedding catturano le relazioni semantiche tra i token. Ad esempio, i token simili in significato sono rappresentati da vettori simili nello spazio degli embedding.

Architettura del modello: Viene definita l'architettura del modello di apprendimento automatico che verrà utilizzato. Nel caso di GPT-3, l'architettura è un trasformatore (Transformer), che è noto per la sua capacità di catturare le relazioni a lungo raggio nei testi.

Struttura del modello: Il modello GPT-3 ha più strati (layer) e ciascun strato contiene diverse unità di calcolo chiamate neuroni. Ogni neurone riceve input dagli embedding dei token e calcola un'uscita basata su pesi appresi durante il training.

Feedforward e backpropagation: Durante il training, i dati vengono passati attraverso il modello in avanti (feedforward). L'output del modello viene confrontato con l'output desiderato e viene calcolato un "errore". Questo errore viene poi propagato all'indietro attraverso il modello (backpropagation), regolando i pesi dei neuroni in modo che l'errore diminuisca progressivamente.

Ottimizzazione: Un algoritmo di ottimizzazione, come l'ottimizzazione del gradiente stocastico (SGD), viene utilizzato per regolare i pesi dei neuroni in modo da ridurre l'errore durante il backpropagation. Questo processo si ripete iterativamente per molti cicli (epoche) sui dati di addestramento.

Regolarizzazione: Per prevenire l'overfitting (adattamento eccessivo ai dati di addestramento), vengono utilizzate tecniche di regolarizzazione come la riduzione del tasso di apprendimento, l'eliminazione casuale e la normalizzazione del batch.

Fine-tuning e validazione: Dopo un certo numero di iterazioni di addestramento, il modello può essere sottoposto a una fase di "fine-tuning" su dati di validazione separati. Questo aiuta a ottimizzare ulteriormente le prestazioni del modello e ad evitare l'overfitting.

Valutazione e test: Una volta che il modello è stato addestrato e sottoposto a fine-tuning, viene valutato su dati di test separati. Questi dati di test non sono stati utilizzati in nessuna fase precedente e servono per valutare le prestazioni generali del modello su nuovi dati.

Iterazione e ottimizzazione: In base alle prestazioni del modello sui dati di test, è possibile apportare regolazioni e miglioramenti all'architettura, all'ottimizzazione e ad altri iperparametri. Questo processo può essere iterato più volte per raggiungere le prestazioni desiderate.

Questi passaggi sono quelli che comunemente vengono utilizzati per allenare una qualsiasi rete neurale, tuttavia ci sono alcune sfumature specifiche che rendono l'addestramento di LLMs un processo particolarmente complesso e interessante:


Dimensioni dei dati e del modello: L'ampiezza dei dati di addestramento e la complessità dell'architettura del modello sono notevoli nei LLMs come GPT-3. Questo richiede una grande quantità di risorse computazionali, tra cui potenti GPU o addirittura TPUs, insieme a soluzioni di calcolo distribuito per accelerare l'addestramento.

Dimensioni del vocabolario: GPT-3 ha un vocabolario estremamente ampio, che include decine di migliaia di token unici. Gestire un vocabolario così grande richiede una tokenizzazione e una gestione specifiche per garantire prestazioni efficienti durante l'addestramento e l'elaborazione successiva.

Generazione del testo: A differenza di alcuni altri tipi di modelli, come quelli di classificazione o segmentazione, i LLMs sono spesso utilizzati per la generazione di testo continuo. Questo richiede un'attenzione particolare alle strategie di addestramento e regolarizzazione per evitare problemi come l'elaborazione di testo senza senso o ripetitivo.

Transfer Learning: I LLMs spesso sfruttano il transfer learning. Vengono preaddestrati su grandi quantità di dati testuali e poi finetunati su dataset specifici o per compiti specifici. Questo approccio consente ai modelli di apprendere rappresentazioni linguistiche generali prima di essere adattati a compiti più specifici.

Rischio etico e contenuto: Dato che i LLMs possono generare testo in modo autonomo, è importante considerare il rischio di generazione di contenuti inappropriati, offensivi o fuorvianti. Questa è una sfida etica che va oltre il semplice addestramento del modello e richiede meccanismi di controllo e moderazione.

Adeguata rappresentazione semantica: La capacità di un LLM di comprendere e generare testo in modo coerente e significativo richiede un'architettura sofisticata come il Transformer, in grado di catturare relazioni a lungo raggio nei testi.

Scelta dell'architettura: Anche se il Transformer è l'architettura predominante per i LLMs, ci sono diverse varianti e miglioramenti, come il GPT-3, che utilizza un modello autoregressivo con attenzione multipla. La scelta dell'architettura e dei suoi parametri può influenzare le prestazioni e la complessità dell'addestramento.

In breve, sebbene il processo di addestramento dei LLMs sia simile a quello di altri modelli di deep learning, le dimensioni, la complessità e le sfumature specifiche dei LLMs portano a sfide uniche e richiedono approcci innovativi per ottenere prestazioni superiori nei compiti di elaborazione del linguaggio naturale.

