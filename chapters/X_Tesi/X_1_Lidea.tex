Un sistema di information retrieval (IR) è progettato per recuperare informazioni rilevanti da un vasto insieme di dati in risposta a una specifica query posta dall'utente. 
Dal 30 novembre 2022, OpenAI con il suo ChatGPT ha rivoluzionato la concenzione che c'era nell'opinione pubblica riguardo l'intelligenza artificiale e il suo potenziale. Anche a livello accademico c'è stato un grande interesse sui risultati ottenuti da OpenAI, e questo ha portato a un'accelerazione della ricerca e dello sviluppo di Large Language Models (LLM).
Questi modelli sono in grado di generare testi coerenti e pertinenti, e sono in grado di rispondere a domande poste in linguaggio naturale. 
Queste caratteristiche potrebbero essere utilizzate per migliorare i sistemi di information retrieval, consentendo agli utenti di interagire con i motori di ricerca in modo più naturale e ottenere risultati più pertinenti.
Qui di seguito, si elencano i vantaggi dell'usare un LLM in questo contesto:
\begin{itemize}
    \item Ricerca basata sulla comprensione del linguaggio: 
    A differenza dei tradizionali motori di ricerca che fanno corrispondenze di parole chiave,
     un LLM può comprendere meglio il significato implicito e il contesto delle richieste degli utenti.
      Può analizzare e interpretare in modo più accurato le domande poste in linguaggio naturale, 
      contribuendo a fornire risultati di ricerca più pertinenti.
    \item Query estese e complesse: GPT-3.5 e simili LLM possono elaborare query complesse e fornire risposte estese. Ciò significa che un sistema di information retrieval basato su LLM potrebbe non solo restituire link a pagine web, ma anche presentare risposte complete alle domande dell'utente, consentendo una comprensione più approfondita del contenuto.
    \item 
    Personalizzazione dell'esperienza di ricerca: Un LLM può apprendere le preferenze e le esigenze dell'utente nel tempo, adattando i risultati delle ricerche in base ai suoi interessi e comportamenti passati. Ciò potrebbe migliorare l'esperienza dell'utente e rendere i risultati di ricerca più pertinenti.
    \item Ricerca semantica: Grazie alla sua capacità di comprendere il significato del testo, un LLM può identificare correlazioni e relazioni semantiche tra diverse fonti di informazione. Questo potrebbe portare a una migliore scoperta di contenuti pertinenti che potrebbero essere stati trascurati da sistemi di ricerca tradizionali.
    \item 
    Generazione di estratti e riassunti: Un LLM potrebbe estrarre automaticamente le parti più rilevanti e informative di un documento o una fonte e presentarle come riassunto. Questo aiuterebbe gli utenti a ottenere una panoramica veloce delle informazioni senza dover leggere l'intero contenuto.

    \item Ricerca multilingue e comprensione interlinguistica: Un LLM addestrato in più lingue potrebbe consentire una ricerca efficace in diverse lingue, agevolando la scoperta di contenuti in tutto il mondo.

    \item 
    Affinamento dei risultati: Un LLM potrebbe interagire con l'utente per affinare i risultati di ricerca attraverso domande di follow-up, assicurandosi di comprendere meglio le sue intenzioni e fornendo risultati più mirati.
    
\end{itemize}