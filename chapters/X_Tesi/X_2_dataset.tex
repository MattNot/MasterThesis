Il dataset contiene 33 manuali e libri di testo scolastici, con una media di 226 pagine, pubblicati tra il 1843 e il 1918 in Canada, Stati Uniti d'America e Gran Bretagna.

La maggior parte dei libri parlano di storia e geografia con alcune eccezioni che parlano di ortografia e grammatica tedesca e inglese.

Essendo libri molto datati, la lingua utilizzata, l'inglese, non è quella attuale, ma una versione più antica e formale.

Questa formalità si riscontra anche nella forma che viene utilizzata, in alcuni libri le lettere iniziali dei paragrafi sono molto grandi e decorate, in altri sono in grassetto, in altri ancora sono in corsivo.

Le caratteristiche riportate sopra rendono complicata la creazione di un sistema di Information Retrieval classico che 
sia in grado di restituire risultati pertinenti anche perché le query espresse non sono strettamente nel linguaggio utilizzato nei libri.
Grazie alle capacità linguistiche e di correzione degli errori ortografici dei modelli di LLM è possibile superare questi problemi senza preoccuparsi troppo di ingegnerizzare il dataset da un punto di vista linguistico.
