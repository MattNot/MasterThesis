
\paragraph{Sommario} 
\textit{
    In questa tesi è proposta la realizzazione di un sistem di information Retrieval per la ricerca di documenti in un archivio di testi.
    Il sistema è stato realizzato in python e utilizza degli embeddings vettoriali per la rappresentazione dei documenti e delle query. 
    Il sistema è stato testato su un archivio di documenti scolastico in formato PDF
 }

\section*{Introduzione}
Nel presente lavoro di tesi, si esamina l'applicazione delle più recenti innovazioni nel campo del deep learning e del Natural Language Processing (NLP) nell'ambito della creazione di un sistema di Information Retrieval (IR), dimostrando il potenziale offerto da tali tecnologie. Attraverso l'implementazione di una proof of concept (PoC), vengono esplorate le possibilità offerte da queste tecnologie allo stato dell'arte, sottolineando come l'adozione delle stesse possa rivoluzionare l'approccio tradizionale all'IR.

La PoC utilizza le risorse e le librerie più avanzate disponibili al momento della sua realizzazione, tra cui spiccano l'utilizzo di librerie come langchain e le API fornite da OpenAI, accelerando notevolmente il processo di sviluppo rispetto alle metodologie tradizionali impiegate nella creazione di sistemi di IR convenzionali. Questo dimostra l'efficacia delle nuove tecnologie e sottolinea l'ampio potenziale di applicazione in contesti reali, tra cui enti di ricerca, biblioteche e aziende con vasti archivi documentali.

Tuttavia, è fondamentale mantenere un approccio critico ed etico nei confronti di queste tecnologie emergenti, nonostante l'entusiasmo crescente dell'opinione pubblica per sistemi come ChatGPT e altre soluzioni basate su NLP. È importante riconoscere che queste innovazioni, se da un lato possono aprire nuove opportunità e migliorare l'accesso alle informazioni, dall'altro possono comportare risvolti etici e sociali che richiedono una valutazione attenta e responsabile.