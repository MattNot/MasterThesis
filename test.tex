\chapter[Gli IDE per ASP]{Gli IDE per ASP}

IDE è un acronimo che sta per Integrated development environment: 

Un sistema che permette di gestire in modo semplice la progettazione, lo sviluppo e la release di un progetto software.

Di seguito alcuni IDE per l'ASP presenti in letteratura.

\section{SeaLion}
SeaLion\cite{SEALION} è un progetto open source mantenuto dal Knowledge-Based System Group dell'università di Vienna.

Il sistema è un plugin per il più noto IDE general-purpose Eclipse.
Le caratteristiche principali sono:
\begin{enumerate}
	\item Highlight della sintassi
	\item Controllo sintattico dopo qualche secondo di inattività
	\item Supporto per tools di terze parti
	\item Visualizzazione grafica dell'output di una esecuzione
\end{enumerate}

In seguito è stato esteso con il plugin Ouroboros che permette di fare debugging di encoding ASP.

Purtroppo il progetto non riceve aggiornamenti dal 2007.

\begin{figure}[H]
	\centering
	\caption{Schermata principale di SeaLion}
\end{figure}

\section{ASPIDE}

ASPIDE\cite{ASPIDE} supporta l'intero ciclo di vita di un progetto basato sull'ASP.
Il team che sviluppa ASPIDE fa riferimento al Dipartimento di Matematica e Informatica dell'Università della Calabria e allo spin-off della stessa, DLVSYSTEM srl.

Le caratteristiche principali sono:
\begin{enumerate}
	\item Strutturazione dei file in workspace
	\item Highlight della sinstassi
	\item Controllo della sintassi in tempo reale
	\item Quick fix sugli errori sintattici e semantici
	\item Completamento automatico
	\item Plugin per la riscrittura delle regole
	\item Rappresentazione grafica del grafo delle dipendenze
	\item Supporto per unit testing
	\item Integrazione con il sistema DLV$^{DB}$
\end{enumerate}
Inoltre ASPIDE supporta la creazione di plugin di terze parti.
\begin{figure}[H]
	\centering
	\caption{Pagina iniziale di ASPIDE}
\end{figure}

\section{LoIDE}

LoIDE è un progetto open source mantenuto dal Dipartimento di Matematica e Informatica dell'Università della Calabria, nasce come lavoro di tesi di uno studente ed è ancora in sviluppo.
Il sistema si presenta come un editor di testo web con alcune funzionalità che aiutano la prototipazione di programmi ASP.
Le principali caratteristiche sono:
\begin{enumerate}
	\item Highlight sintassi
	\item Personalizzazione dei colori dell'interfaccia
	\item Highlight degli atomi in output 
	\item Import e Export delle opzioni in file JSON
	\item Autocompletamento
\end{enumerate}

\begin{figure}[H]
	\centering
	\caption{Pagina principale di LoIDE}
\end{figure}