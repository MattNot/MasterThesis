\title{SchoolGPT: Information retrieval with LLM}

\author{Some Authors\cite{Author1}}

\newcommand{\abstractText}{\noindent
This paper presents the development of an Information Retrieval system for document search within a text archive. The system has been implemented in Python and employs vector embeddings for document and query representations. Extensive testing was conducted using a collection of educational documents in PDF format.
}

%%%%%%%%%%%%%%%%%
% Configuration %
%%%%%%%%%%%%%%%%%

\documentclass[12pt, a4paper, twocolumn]{article}
\usepackage{xurl}
\usepackage[super,comma,sort&compress]{natbib}
\usepackage{abstract}
\renewcommand{\abstractnamefont}{\normalfont\bfseries}
\renewcommand{\abstracttextfont}{\normalfont\small\itshape}
\usepackage{lipsum}

%%%%%%%%%%%%%%
% References %
%%%%%%%%%%%%%%

% If changing the name of the bib file, change \bibliography{test} at the bottom
\begin{filecontents}{test.bib}

@misc{LinkReference1,
  title        = "Link Title",
  author       = "Link Creator(s)",
  howpublished = "\url{https://example.com/}",
}

@misc{Author1,
  author       = "LastName, FirstName",
  howpublished = "\url{mailto:email@example.com}",
}

@article{ArticleReference1,
  author  = "Lastname1, Firstname1 and Lastname2, Firstname2",
  title   = "Article title",
  year    = "Year",
  journal = "Journal name",
  note    = "\url{https://dx.doi.org/...}",
}

\end{filecontents}

% Any configuration that should be done before the end of the preamble:
\usepackage{hyperref}
\hypersetup{colorlinks=true, urlcolor=blue, linkcolor=blue, citecolor=blue}

\begin{document}

%%%%%%%%%%%%
% Abstract %
%%%%%%%%%%%%

\twocolumn[
  \begin{@twocolumnfalse}
    \maketitle
    \begin{abstract}
      \abstractText
      \newline
      \newline
    \end{abstract}
  \end{@twocolumnfalse}
]

%%%%%%%%%%%
% Article %
%%%%%%%%%%%

\section{Introduction}

We'll explore the application of cutting-edge advancements in the fields of deep learning and Natural Language Processing (NLP) within the realm of Information Retrieval (IR) systems, showcasing the potential offered by these technologies. Through the implementation of a proof of concept (PoC), we delve into the possibilities these state-of-the-art technologies bring to the forefront, emphasizing how their adoption has the potential to revolutionize the conventional approach to IR.

The PoC harnesses the most advanced resources and libraries available at the time of its development, prominently featuring the use of libraries such as langchain and APIs provided by OpenAI, significantly expediting the development process compared to traditional methodologies employed in the creation of conventional IR systems. This underscores the efficacy of these new technologies and underscores their wide-ranging applicability in real-world contexts, including research institutions, libraries, and companies with extensive document archives.

\section{State of art}

The field of information retrieval is one of the most dynamic and new technologies are presented at every conference.


%%%%%%%%%%%%%%
% References %
%%%%%%%%%%%%%%

\nocite{*}
\bibliographystyle{plain}
\bibliography{test}

\end{document}